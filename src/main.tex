\documentclass[conference]{IEEEtran}
\IEEEoverridecommandlockouts
% The preceding line is only needed to identify funding in the first footnote. If that is unneeded, please comment it out.
\usepackage{cite}
\usepackage{amsmath,amssymb,amsfonts}
\usepackage{algorithmic}
\usepackage{graphicx}
\usepackage[utf8]{inputenc}
\usepackage[portuguese]{babel}
\usepackage{textcomp}
\usepackage{xcolor}
\usepackage{pgfplots}
\usepackage{hyperref}
\usepackage{float}
\pgfplotsset{compat=1.4}
\usepgfplotslibrary{statistics}
\def\BibTeX{{\rm B\kern-.05em{\sc i\kern-.025em b}\kern-.08em
    T\kern-.1667em\lower.7ex\hbox{E}\kern-.125emX}}

    \hypersetup{
    pdfsubject={Estudo de hierarquia de memória},
    pdfcreator={GLR, LSO, THN},
    colorlinks=true,       		% false: boxed links; true: colored links
    linkcolor=black,          	% color of internal links
    citecolor=black,        		% color of links to bibliography
    filecolor=black,      		% color of file links
    urlcolor=black,
    bookmarksdepth=4
}


\begin{document}

\title{Estudo de hierarquia de memória}

\author{\IEEEauthorblockN{1\textsuperscript{st} Gustavo Lopes Rodrigues}
\IEEEauthorblockA{\textit{Instituto de Ciências Exatas e Informática} \\
\textit{Pontifícia Universidade Católica de Minas Gerais (PUC-MG)}\\
Belo Horizonte, Brasil \\
gustavolr@gmail.com}
\and
\IEEEauthorblockN{2\textsuperscript{nd} Homenique Vieira Martins}
\IEEEauthorblockA{\textit{Instituto de Ciências Exatas e Informática} \\
\textit{Pontifícia Universidade Católica de Minas Gerais (PUC-MG)}\\
Belo Horizonte, Brasil \\
Homenique.T@gmail.com}
\and
\IEEEauthorblockN{3\textsuperscript{rd} Lucas Santiago de Oliveira}
\IEEEauthorblockA{\textit{Instituto de Ciências Exatas e Informática} \\
\textit{Pontifícia Universidade Católica de Minas Gerais (PUC-MG)}\\
Belo Horizonte, Brasil \\
lu.santi.oliveira@gmail.com}
\and
\IEEEauthorblockN{4\textsuperscript{th} Thiago Henriques Nogueira}
\IEEEauthorblockA{\textit{Instituto de Ciências Exatas e Informática} \\
\textit{Pontifícia Universidade Católica de Minas Gerais (PUC-MG)}\\
Belo Horizonte, Brasil \\
thiagohnogueira01@gmail.com}
}

\maketitle

\begin{abstract}
   Memória Cache em computadores tem vários algoritmos de 
  substituição de dados, dois desses algoritmos são \emph{FIFO}( First In, First Out )
  e \emph{LRU}( Least Recently Used ), decidimos estudar a eficiência destes dois algoritmos.
  Nossa ideia antecedente é de que o LRU fosse alcançar um desempenho superior ao FIFO, com o 
  aumento do número de instruções e o aumento progressivo da Cache. Foram analisados
  em vários testes com quantidade de instruções diferentes e em hardwares,
  variados visando avaliar como esses algoritmos atuariam em tais situações.
  Desta forma, poderíamos ter uma visão clara de que tanto a modificação
  do hardware quanto das instruções que executam sobre eles têm um impacto significativo
  na eficiência da leitura e escrita de dados da RAM para a cache ou de volta para a memória
  principal.
\end{abstract}

\begin{IEEEkeywords}
Cache, algoritmos, FIFO, LRU, Tamanho Cache
\end{IEEEkeywords}

\section{Introdução}

%Espera-se que nesta seção, seja escrita uma contextualização inicial 
%para situar o leitor. Ou seja, saber em qual área ele está "pisando".
%No contexto discutido, é apresentado um problema em aberto (um problema 
%pode ser escrito como uma pergunta), em seguida um objetivo para "responder" 
%ao problema descrito. Depois, a contribuição científica é descrita, e por fim, 
%um parágrafo para descrever a organização do artigo é feito.

A Cache é um dispositivo de acesso rápido, que fica localizado dentro do processador,
com a intenção de reduzir o acesso do processador à memória principal, que demanda um
tempo de acesso muito superior à cache. Criando uma referência da localidade do dado na 
memória principal, até mesmo gravando esses dados, porém existe um limite de quantos 
dados podem ser armazenados para isso é necessário uma política de armazenamento para 
reocupar espaço quando necessário.

Desde a invenção dos computadores, a busca por otimizar os processos que são executados 
nos hardwares se tornou foco, considerando que simples ajustes podem render grandes 
ganhos de desempenho. Dessa forma, um algoritmo eficiente na leitura e escrita de 
dados na cache é de suma importância. 

Com isso em mente, pensamos em políticas de substituição clássicos, LRU e o FIFO, para
averiguar qual possui melhor desempenho em diferentes cenários. Começamos pensando em 
aumentar gradativamente o número de instruções que acessam as palavras na RAM para serem 
escritas na cache. Dessa forma, seria simples visualizar como o número de instruções
influencia no tempo de acesso aos dados. Além disso, pensamos também em vários cenários 
com caches de diferentes tamanhos, desde de o mínimo de espaço possível até uma cache 
de mesmo tamanho que a memória RAM.

\section{Trabalhos Correlatos}

%Nesta seção, artigos relacionados à simulações de memória são apresentados.
%Por exemplo, para cada artigo, um parágrafo. Ao final, um parágrafo que menciona
%a diferença entre o artigo em questão e os artigos correlatos.

A partir de James E. Smith e James R. Goodman\cite{b1}, tanto algoritmos de LRU quanto de FIFO para
uma cache possuem o pior desempenho possível quando comparados a qualquer outro algoritmo. 
O exemplo principal deles foi mostrar como mapeamento aleatório consegue ser várias vezes mais eficaz, 
uma vez que programas que precisam reescrever muitas vezes a cache e remover um valor que será reutilizado 
em um futuro muito próximo.

Considerando que é um artigo escrito em 1985, muitas coisas mudaram e não podemos usar mais esses resultados
como verdadeiros em situações do dia-a-dia. Tanto o LRU quanto o FIFO se mostram ineficientes quando
os programas se tornam maiores do que o tamanho de toda a cache do processador (resultado apresentado pelo artigo
deles). Hoje, temos caches muito maiores do que naquela época, com isso, o principal foco da pesquisa do Smith e 
do Goodman diminuindo o seu valor nos dias de hoje.Pois eles estavam considerando caches extremamente pequenas, na maior 
parte das vezes inferior ao tamanho de aplicações comuns do cotidiano.

Por outro lado, o artigo \cite{b2}, por ser de 2008 poderia ser mais promissor em apresentar 
dados que ajudariam em nossa pesquisa, porém não foi isso que aconteceu. A pesquisa da Universidade de 
Rennes faz também análise da memória Cache em relação as políticas de substituição. Entretanto, as políticas 
utilizadas(fora a LRU) são diferentes e não há a utilização da FIFO. Além de que, o objetivo da pesquisa é de usar 
dados teóricos para melhorar um método de análise de instrução estática de cache usando Pseudo LRU e a política 
aleatória de troca. E mesmo assim, o próprio artigo provou que métodos não LRU demonstraram uma significante perca de precisão.

\section{Metodologia ou Proposta de Arquiteturas}

%A metodologia utilizada no artigo consiste no estudo de artigos que se correlacionam com o tema estudado, 
%como também a utilização e desenvolvimento de arquiteturas feitas no simulador Amnesia.
  Para a elaboração de teste utilizamos o simulador-Amnésia para elaborar algumas
arquiteturas (tabelas \ref{tab1} e \ref{tab2}) que a partir delas executamos os testes (tabela \ref{tab3})
\begin{table}[H]
  \caption{características gerais da arquitetura i}
    \centering
      \begin{tabular}{|c|c|c|c|}
          \hline
          \textbf{Especificações} & \multicolumn{3}{|c|}{\textbf{Partes da Arquitetura}} \\
          \cline{2-4} 
          \textbf{da Arquitetura} & \textbf{\textit{Processador}}& \textbf{\textit{CPU}}& \textbf{\textit{Trace}} \\
          \hline
          Tamanho da palavra & --- & 4 & 4 \\
          \hline
          processorContains & 0 & --- & --- \\
          \hline
          Ciclos por escrita & 0 & --- & --- \\
          \hline
      \end{tabular}
      \label{tab1}
\end{table}

\begin{table}[H]
  \caption{características gerais da arquitetura ii}
  \centering
      \begin{tabular}{|c|c|c|}
          \hline
          \textbf{Especificações} & \multicolumn{2}{|c|}{\textbf{Partes da Arquitetura}} \\
          \cline{2-3} 
          \textbf{da Arquitetura} & \textbf{\textit{Memória Principal}}& \textbf{\textit{Cache}} \\
          \hline
          Tamanho da linha / bloco & 1 & 1  \\
          \hline
          Ciclos por leitura & 1 & 1  \\
          \hline
          Ciclos por escrita & 2 & 2  \\
          \hline
          Tempo do ciclo & 10 & 1  \\
          \hline
          Tamanho da memória & 16 & {$^{\mathrm{*}}$}2  \\
          \hline
          Associatividade & --- & 2  \\
          \hline
          Politica de escrita & --- & Write-Through  \\
          \hline
          Politica de substituição & --- & {$^{\mathrm{*}}$}FIFO  \\
          \hline
          \multicolumn{3}{l}{$^{\mathrm{*}}$ Os valores com o asterisco foram os valores modificados.}
      \end{tabular}
      \label{tab2}
\end{table}

As tabelas acima mostram a arquitetura inicial utilizada no projeto para o 
desenvolvimento dos diversos cenários que foram originados.    

\begin{table}[H]
  \caption{Dados modificados}
  \centering
      \begin{tabular}{|c|c|c|}
          \hline
          \textbf{Especificações} & \multicolumn{2}{|c|}{\textbf{Valores}} \\
          \cline{2-3} 
          \textbf{da Arquitetura} & \textbf{Tamanho da Memoria} & \textbf{Politica de Substituição} \\
          \hline
          Cenário 1 & 2 & FIFO \\
          \hline
          Cenário 2 & 2 & LRU \\
          \hline
          Cenário 3 & 4 & FIFO\\
          \hline
          Cenário 4 & 4 & LRU\\
          \hline
          Cenário 5 & 8 & FIFO\\
          \hline
          Cenário 6 & 8 & LRU\\
          \hline
          Cenário 7 & 16 & FIFO\\
          \hline
          Cenário 8 & 16 & LRU\\
          \hline
          %\multicolumn{3}{l} {Sample of a Table footnote.}
      \end{tabular}
      \label{tab3}
\end{table}

Ao modificar o valor do tamanho da memória e da política de substituição da cache,
é possível criar 8 cenários distintos para se avaliar o impacto do tamanho da cache
no desempenho dos metodos de substituição. Sendo assim, com os cenários propostos
foram realizados diversos testes com o intuito de analisar a Localidade Temporal e 
Localidade Espacial nas arquiteturas.

\section{Avaliação dos Resultados}

%Nesta seção, os resultados são apresentados em gráficos. Para cada figura,
%importante lembrar que deve ter texto com explicação.
%Caso a metodologia de avaliação não tenha sido descrita em uma seção específica,
%esta pode ser descrita no início desta seção.

Nos gráficos subsequentes, há uma demonstração de como a memória cache se comporta 
ao ser usada com um grande conjunto de instruções de \emph{load} e \emph{store} 
(maior que sua capacidade) em cenários com caches de 2, 4, 8 e 16 linhas.
Para fazer esses testes utilizamos um conjunto de 10, 50, 100, 250, 500, 750, 1.000 e 10.000 
instruções.

\subsection{Gráficos de hit rate por número de instruções}

  \begin{figure}[H]
    \centering
    \caption{\textit{\textbf{Cache com algoritmo de substituição FIFO}}}
  \begin{tikzpicture}

    \begin{axis}[
      width =\linewidth - 21,
      xlabel=Quantidade de instrução,
      ylabel=\textbf{Hit rate},
      domain = 1:10000,
      xmin=10, xmax=10000,
      ymin=0, ymax=1,
      xmode = log,
      log basis x={10},
      clip=false,
      every axis plot/.append style={thick},
      ymajorgrids=true,
      legend  style={at={(0.5 ,-0.20)},
      anchor=north,legend  columns =-1},
    ]
    \addplot [mark=*] coordinates {
      (10, 0.1)
      (50, 0.12)
      (100, 0.12)
      (250, 0.168)
      (500, 0.196)
      (750, 0.19733334)
      (1000, 0.207)
      (10000, 0.1947)
    };
    \addplot [mark=triangle] coordinates {
      (10, 0.2)
      (50, 0.22)
      (100, 0.28)
      (250, 0.356)
      (500, 0.374)
      (750, 0.36666667)
      (1000, 0.381)
      (10000, 0.3913)
   };
   \addplot [mark=square] coordinates {
      (10, 0.3)
      (50, 0.6)
      (100, 0.71)
      (250, 0.788)
      (500, 0.782)
      (750, 0.7733333)
      (1000, 0.786)
      (10000, 0.7983)
   };
   \addplot [solid] coordinates {
      (10, 0.3)
      (50, 0.8)
      (100, 0.9)
      (250, 0.96)
      (500, 0.98)
      (750, 0.9866667)
      (1000, 0.99)
      (10000, 0.999)
   };

  \legend{2 linhas, 4 linhas, 8 linhas, 16 linhas}

    \end{axis}
    
  \end{tikzpicture}

\end{figure}

  \begin{figure}[H]
    \caption{\textit{\textbf{Cache com algoritmo de substituição LRU}}}
  \begin{tikzpicture}

    \begin{axis}[
      width =\linewidth - 21,
      xlabel=Quantidade de instrução,
      ylabel=\textbf{Hit rate},
      domain = 1:10000,
      xmin=10, xmax=10000,
      ymin=0, ymax=1,
      xmode = log,
      log basis x={10},
      clip=false,
      every axis plot/.append style={thick},
      ymajorgrids=true,
      legend  style={at={(0.5 ,-0.20)},
      anchor=north,legend  columns =-1},
    ]
    \addplot [mark=*] coordinates {
      (10, 0.1)
      (50, 0.14)
      (100, 0.13)
      (250, 0.172)
      (500, 0.194)
      (750, 0.19733334)
      (1000, 0.205)
      (10000, 0.1945)
    };
    \addplot [mark=triangle] coordinates {
      (10, 0.2)
      (50, 0.26)
      (100, 0.3)
      (250, 0.368)
      (500, 0.374)
      (750, 0.364)
      (1000, 0.383)
      (10000, 0.3928)
   };
   \addplot [mark=square] coordinates {
      (10, 0.3)
      (50, 0.58)
      (100, 0.7)
      (250, 0.788)
      (500, 0.784)
      (750, 0.77066666)
      (1000, 0.788)
      (10000, 0.7961)
   };
   \addplot [solid] coordinates {
      (10, 0.3)
      (50, 0.8)
      (100, 0.9)
      (250, 0.96)
      (500, 0.98)
      (750, 0.9866667)
      (1000, 0.99)
      (10000, 0.999)
   };

  \legend{2 linhas, 4 linhas, 8 linhas, 16 linhas}

    \end{axis}
    
  \end{tikzpicture}

\end{figure}

Os dois gráficos acima foram montados usando os valores que extraímos. 
Nota-se que os resultados foram muito próximos, aos resultados de James 
E. Smith e James R. Goodman\cite{b1}. Considerando que no início, pensavamos que haveria
uma diferença significativa entre os algoritmos LRU e FIFO para substituição de valores na 
cache, achamos que nossos testes estavam errados e que precisariamos refazê-los, olhando 
a semelhança de resultados entre os nossos experimentos e os experimentos do artigo vimos 
que os valores eram válidos.

\subsection{Tempo total de acesso aos dados por quantidade de instruções}

\raggedbottom

\begin{figure}[H]

  \caption{\textit{\textbf{Tempo total de acesso aos dados com algoritmo FIFO}}}
  \begin{tikzpicture}

    \begin{axis}[
      width =\linewidth - 21,
      xlabel=Quantidade de instrução,
      ylabel=\textbf{Tempo de acesso à Cache},
      domain = 1:10000,
      xmin=10, xmax=10000,
      ymin=1, ymax=100000,
      ymode = log,
      log basis y={10},
      xmode = log,
      log basis x={10},
      clip=false,
      every axis plot/.append style={thick},
      ymajorgrids=true,
      legend  style={at={(0.5 ,-0.20)},
      anchor=north,legend  columns =-1},
    ]
    \addplot [mark=*] coordinates {
      (10, 100)
      (50, 490)
      (100, 980)
      (250, 2330)
      (500, 4520)
      (750, 6770)
      (1000, 8930)
      (10000, 90530)
    };
    \addplot [mark=triangle] coordinates {
      (10, 90)
      (50, 440)
      (100, 820)
      (250, 1860)
      (500, 3630)
      (750, 5500)
      (1000, 7190)
      (10000, 70870)
   };
   \addplot [mark=square] coordinates {
      (10, 80)
      (50, 250)
      (100, 390)
      (250, 780)
      (500, 1590)
      (750, 2450)
      (1000, 3140)
      (10000, 30170)
   };
   \addplot [solid] coordinates {
      (10, 80)
      (50, 150)
      (100, 200)
      (250, 350)
      (500, 600)
      (750, 850)
      (1000, 1100)
      (10000, 10100)
   };

  \legend{2 linhas, 4 linhas, 8 linhas, 16 linhas}

    \end{axis}
    
  \end{tikzpicture}

\end{figure}

  \begin{figure}[H]


  \caption{\textit{\textbf{Tempo total de acesso aos dados com algoritmo LRU}}}

  \begin{tikzpicture}

    \begin{axis}[
      width =\linewidth - 21,
      xlabel=Quantidade de instrução,
      ylabel=\textbf{Tempo de acesso a Cache},
      domain = 1:10000,
      xmin=10, xmax=10000,
      ymin=1, ymax=100000,
      ymode = log,
      log basis y={10},
      xmode = log,
      log basis x={10},
      clip=false,
      every axis plot/.append style={thick},
      ymajorgrids=true,
      legend  style={at={(0.5 ,-0.20)},
      anchor=north,legend  columns =-1},
    ]
    \addplot [mark=*] coordinates {
      (10, 100)
      (50, 480)
      (100, 970)
      (250, 2320)
      (500, 4530)
      (750, 6770)
      (1000, 8950)
      (10000, 90550)
    };
    \addplot [mark=square] coordinates {
      (10, 90)
      (50, 420)
      (100, 800)
      (250, 1830)
      (500, 3630)
      (750, 5520)
      (1000, 7170)
      (10000, 70720)
   };
   \addplot [mark=triangle] coordinates {
      (10, 80)
      (50, 260)
      (100, 400)
      (250, 780)
      (500, 1580)
      (750, 2470)
      (1000, 3120)
      (10000, 30390)
   };
   \addplot [solid] coordinates {
      (10, 80)
      (50, 150)
      (100, 200)
      (250, 350)
      (500, 600)
      (750, 850)
      (1000, 1100)
      (10000, 10100)
   };

  \legend{2 linhas, 4 linhas, 8 linhas, 16 linhas}

    \end{axis}
    
  \end{tikzpicture}

\end{figure}

Os dois gráficos anteriores, seguindo a mesma linha do hit rate, mantêm proximidade
dos valores no tempo de acesso dos dados, desde copiar da RAM para a cache
até acessar esse valor no processador. Dentro dos valores que experimentamos, os 
valores se mantiveram bem próximos. O tempo de acesso aos dados se mostrou fora do padrão
apenas no caso da cache com 16 linhas (mesmo tamanho que o processador), teve resultados 
que foram significativos, houve uma grande queda no tempo de acesso dos dados.
Em todos os outros casos, o tempo se manteve muito próximo, mesmo que entre caches de
2 a 8 linhas de tamanho.

\section{Conclusões}

%Nas conclusões, inicia-se com um breve resumo do que foi o trabalho desenvolvido,
%depois faz-se uma discussão acerca dos objetivos alcançados. Ou seja, os objetivos
%apresentados na Introdução casam com os resultados discutidos na seção de Resultados? 
%Quais resultados podem ser destacados? Por fim, há um parágrafo para apresentar possíveis trabalhos futuros.

Após a realização dos testes em todos os cenários propostos, e uma análise dos gráficos
plotados houve um receio de que o ponto da pesquisa não havia sido alcançado, já que 
os resultados se mostraram pouco expressivos.

Porém, após uma análise e estudo dos artigos correlacionados, foi possível observar 
que a política de substituição LRU se mostra de forma pouco significativa mais eficiente
do que a política FIFO. 

\begin{thebibliography}{00}
\bibitem{b1} J. Smith and B. Goodman, ''Instruction Cache Replacement Policies and Organizations'' in \emph{IEEE Transactions on Computers}, vol. C-34, no.3, pp. 234--241, March 1985.
\bibitem{b2} A. Junier, D. Hardy and I. Puaut, ''Impact of instruction cache replacement policy on the tightness of WCET estimation'', in \emph{IRISA}, University of Rennes, [Documento online] , 2008. Disponível em: ReserchGate, \url{https://www.researchgate.net/publication/239761570_Impact_of_instruction_cache_replacement_policy_on_the_tightness_of_WCET_estimation} [Acessado em: 14 de Abril de 2021] 
\bibitem{b3} S. Ajorpaz, E. Garza, S. Jindal and D. Jiménez, ''Exploring Predictive Replacement Policies for Instruction Cache and Branch Target Buffer,'' in \emph{ACM/IEEE 45th Annual International Symposium on Computer Architecture (ISCA)}, Los Angeles, CA, USA, 2018, pp. 519-532
\end{thebibliography}

\end{document}
\documentclass[conference]{IEEEtran}
\IEEEoverridecommandlockouts
% The preceding line is only needed to identify funding in the first footnote. If that is unneeded, please comment it out.
\usepackage{cite}
\usepackage{amsmath,amssymb,amsfonts}
\usepackage{algorithmic}
\usepackage{graphicx}
\usepackage[portuguese]{babel}
\usepackage{textcomp}
\usepackage{xcolor}
\def\BibTeX{{\rm B\kern-.05em{\sc i\kern-.025em b}\kern-.08em
    T\kern-.1667em\lower.7ex\hbox{E}\kern-.125emX}}
\begin{document}

\title{Estudo de hierarquia de memória}

\author{\IEEEauthorblockN{1\textsuperscript{st} Gustavo Lopes Rodrigues}
\IEEEauthorblockA{\textit{Instituto de Ciências Exatas e Informática} \\
\textit{Pontifícia Universidade Católica de Minas Gerais(PUC-MG)}\\
Belo Horizonte, Brasil \\
gustavolr@gmail.com}
\and
\IEEEauthorblockN{2\textsuperscript{nd} Homenique Vieira Martins}
\IEEEauthorblockA{\textit{Instituto de Ciências Exatas e Informática} \\
\textit{Pontifícia Universidade Católica de Minas Gerais(PUC-MG)}\\
Belo Horizonte, Brasil \\
Homenique.T@gmail.com}
\and
\IEEEauthorblockN{3\textsuperscript{rd} Lucas Santiago de Oliveira}
\IEEEauthorblockA{\textit{Instituto de Ciências Exatas e Informática} \\
\textit{Pontifícia Universidade Católica de Minas Gerais(PUC-MG)}\\
Belo Horizonte, Brasil \\
lu.santi.oliveira@gmail.com}
\and
\IEEEauthorblockN{4\textsuperscript{th} Thiago Henriques Nogueira}
\IEEEauthorblockA{\textit{Instituto de Ciências Exatas e Informática} \\
\textit{Pontifícia Universidade Católica de Minas Gerais(PUC-MG)}\\
Belo Horizonte, Brasil \\
thiagohnogueira01@gmail.com}
}

\maketitle

\begin{abstract}
Este aqui é um exemplo de abstract, é aqui 
onde você dá uma prévia sobre o que é o
artigo que será feito
\end{abstract}

\begin{IEEEkeywords}
palavras, chaves
\end{IEEEkeywords}

\section{Introdução}

Espera-se que nesta seção, seja escrita uma contextualização inicial 
para situar o leitor. Ou seja, saber em qual área ele está "pisando".
No contexto discutido, é apresentado um problema em aberto (um problema 
pode ser escrito como uma pergunta), em seguida um objetivo para "responder" 
ao problema descrito. Depois, a contribuição científica é descrita, e por fim, 
um parágrafo para descrever a organização do artigo é feito.

\section{Trabalhos Correlatos}

Nesta seção, artigos relacionados à simulações de memória são apresentados.
Por exemplo, para cada artigo, um parágrafo. Ao final, um parágrafo que menciona
a diferença entre o artigo em questão e os artigos correlatos.

\subsection{Exemplo de subsection}

Como colocar subsection no sistema IEEE

\section{Metodologia ou Proposta de Arquiteturas}

É possível ter duas seções, uma para cada, mas somente uma seção chamada 
"Metodologia" ou "Propostas de Arquiteturas" será aceita. Nesta seção, as 
arquiteturas simuladas são apresentadas, por exemplo, em formato de tabela, 
com características, tais como, tamanho, associatividade, etc.  Pode ter figura
de arquitetura em um formato de diagrama em blocos. Esta seção é importante, 
porque ela explica e descreve o que será simulado, ou seja, os cenários escolhidos 
por vocês para simulações.

\section{Avaliação dos Resultados}

Nesta seção, os resultados são apresentados em gráficos. Para cada figura,
importante lembrar que deve ter texto com explicação.
Caso a metodologia de avaliação não tenha sido descrita em uma seção específica,
esta pode ser descrita no início desta seção.

\section{Conclusões}

Nas conclusões, inicia-se com um breve resumo do que foi o trabalho desenvolvido,
depois faz-se uma discussão acerca dos objetivos alcançados. Ou seja, os objetivos
apresentados na Introdução casam com os resultados discutidos na seção de Resultados? 
Quais resultados podem ser destacados? Por fim, há um parágrafo para apresentar possíveis trabalhos futuros.

\begin{thebibliography}{00}
\bibitem{b1} G. Eason, B. Noble, and I. N. Sneddon, ``On certain integrals of Lipschitz-Hankel type involving products of Bessel functions,'' Phil. Trans. Roy. Soc. London, vol. A247, pp. 529--551, April 1955.
\bibitem{b2} J. Clerk Maxwell, A Treatise on Electricity and Magnetism, 3rd ed., vol. 2. Oxford: Clarendon, 1892, pp.68--73.
\bibitem{b3} I. S. Jacobs and C. P. Bean, ``Fine particles, thin films and exchange anisotropy,'' in Magnetism, vol. III, G. T. Rado and H. Suhl, Eds. New York: Academic, 1963, pp. 271--350.
\bibitem{b4} K. Elissa, ``Title of paper if known,'' unpublished.
\bibitem{b5} R. Nicole, ``Title of paper with only first word capitalized,'' J. Name Stand. Abbrev., in press.
\bibitem{b6} Y. Yorozu, M. Hirano, K. Oka, and Y. Tagawa, ``Electron spectroscopy studies on magneto-optical media and plastic substrate interface,'' IEEE Transl. J. Magn. Japan, vol. 2, pp. 740--741, August 1987 [Digests 9th Annual Conf. Magnetics Japan, p. 301, 1982].
\bibitem{b7} M. Young, The Technical Writer's Handbook. Mill Valley, CA: University Science, 1989.
\end{thebibliography}
\vspace{12pt}
\color{red}
IEEE conference templates contain guidance text for composing and formatting conference papers. Please ensure that all template text is removed from your conference paper prior to submission to the conference. Failure to remove the template text from your paper may result in your paper not being published.

\end{document}